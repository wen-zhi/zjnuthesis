\chapter{常见问题}

\section{模板的架构设计}

模板的定义代码位于 \texttt{style} 文件夹中。其中:
\begin{itemize}
\item \texttt{zjnuthesis.cls} 用于定义模板的风格,载入了模板所必须依赖的宏包
\item \texttt{customize.tex} 用于存放自定义的命令、放置用户所需的宏包
\end{itemize}

模板本身所引入的宏包已能够适应绝大多数场景,无需引入额外的宏包。如需引入宏包,推荐放入 \texttt{customize.tex} 中以保持主文件 \texttt{thesis.tex} 的整洁。

\section{论文标题很长时怎么办}

论文标题超过 15 个字时,封面的标题风格可能会不遂人意。能不能让两行都居中显示?

由于难以自动地在合适的地方将标题划分为两个部分,比如:

\begin{table}[h]
\centering\bfseries\zihao{4}\heiti
   \begin{tabularx}{0.6\linewidth}{l X<{\centering\arraybackslash}}
       题{\quad}目: & \uline{\hfill 唐僧在女儿国抒怀并看着女 \hfill} \\
       & \uline{\hfill 儿国王的眼睛 \hfill}
   \end{tabularx}
\end{table}

其效果显然不如:

\begin{table}[h]
\centering\bfseries\zihao{4}\heiti
   \begin{tabularx}{0.6\linewidth}{l X<{\centering\arraybackslash}}
       题{\quad}目: & \uline{\hfill 唐僧在女儿国抒怀并看着 \hfill} \\
       & \uline{\hfill 女儿国王的眼睛 \hfill}
   \end{tabularx}
\end{table}

因此,本模板考虑让用户自行对标题进行划分。即当标题长度大于 15 个字时,可打开 \texttt{pages} 文件夹下的 \texttt{cover.tex} 手动填充标题。具体做法为:

1) 首先注释掉 24-31 行中的使用一行标题的代码 (TeXstudio 中的快捷键为 `\texttt{Ctrl} + \texttt{T}')。

2) 然后解除 34-42 行中的注释 (TeXstudio 中的快捷键为 `\texttt{Ctrl} + \texttt{U}'),并在给出的示范位置中分别填充两行标题的内容。

\section{定理、证明环境怎么定义}

模板并未包含定理环境,如有需要可在 \texttt{./style/customize.tex} 中自定义。实际上,\texttt{customize.tex} 提供了一个\textbf{定义环境}的样例(第 14 行):
\begin{verbatim}
\newtheorem{definition}{定义}[chapter]
\end{verbatim}

定义新的定理环境其实非常简单,下面给出三个示例,若需使用,将其复制到 \texttt{customize.tex} 中即可:

\begin{verbatim}
\usepackage{amsthm}  % 提供证明环境 (i.e. \begin{proof}\end{proof})
\newtheorem{theorem}{定理}[chapter]      % 定理环境
\newtheorem{proposition}{命题}[chapter]  % 命题环境
\newtheorem{lemma}{引理}[chapter]        % 引理环境
\end{verbatim}

如需其它类定理环境,可按照上述模式定义即可。
