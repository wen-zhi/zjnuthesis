
% 中文摘要
\begin{abstract-zh}

君子既得其养,又好其辨也。所谓辨者,贵贱有等,长少有差,贫富轻重皆有称也。故天子大路越席,所以养体也;侧载臭茝,所以养鼻也;前有错衡,所以养目也;和鸾之声,步中武象,骤中韶濩,所以养耳也;龙旂九斿,所以养信也;寝兕持虎,鲛韅弥龙,所以养威也。故大路之马,必信至教顺,然后乘之,所以养安也。孰知夫出死要节之所以养生也。孰知夫轻费用之所以养财也,孰知夫恭敬辞让之所以养安也,孰知夫礼义文理之所以养情也。

人苟生之为见,若者必死;苟利之为见,若者必害;怠惰之为安,若者必危;情胜之为安,若者必灭。故圣人一之於礼义,则两得之矣;一之於情性,则两失之矣。故儒者将使人两得之者也,墨者将使人两失之者也。是儒墨之分。

\end{abstract-zh}

% 中文关键词 (不要插入空白行)
\begin{keywords-zh}
恭敬辞让;礼义文理;君子好辨;儒墨之分
\end{keywords-zh}
